% Command definitions

% Paragraph breaks
\newcommand{\br}{\\[2mm]}
\newcommand{\brbr}{\\[4mm]}

% Definitions - Math environment
\newcommand{\RA}{\Rightarrow} % shorthands for implication and equivalence arrows
\newcommand{\LA}{\Leftarrow}
\newcommand{\LongRA}{\Longrightarrow}
\newcommand{\LongLA}{\Longleftarrow}
\newcommand{\LRA}{\Leftrightarrow}
\newcommand{\LongLRA}{\Longleftrightarrow}

\newcommand{\ceq}{\coloneqq}

\newcommand{\Nat}{\mathbb{N}}

% Values
\newcommand{\True}{\textsf{True}}
\newcommand{\False}{\textsf{False}}

% Types
\newcommand{\Int}{\textsf{Int}}
\newcommand{\Bool}{\textsf{Bool}}
\newcommand{\Char}{\textsf{Char}}


% Gray highlighting
\newcommand{\gbox}[2][gray!20]{\mathpalette{\highlightwithstyle[#1]}{#2}}
\newcommand{\highlightwithstyle}[3][gray!20]{
    \begingroup                         %% <- limit scope of \box0 and \fboxsep assignment
    \sbox0{$\mathsurround 0pt #2#3$}%   %% <- typeset content in box 0
    \setlength{\fboxsep}{0.8pt}          %% <- set (smaller) framebox margins
    \sbox2{\hspace{-0.8pt}%              %% <- create box 2, undo margin
        \colorbox{#1}{\usebox0}%        %% <- print the contents of box 0 in a \colorbox
    }%
    \dp2=\dp0 \ht2=\ht0 \wd2=\wd0       %% <- set dimensions of box 2 to match box 0
    \box2                               %% <- print box 2
    \endgroup                           %% <- revert old definitions of the boxes and \fboxsep
}

% Font styles
\newcommand{\textsfit}[1]{\text{\sffamily\itshape#1}}
\newcommand{\fatcal}[1]{\bm{\mathcal{#1}}}

\newcommand{\keyword}[1]{\textbf{#1}}
\newcommand{\keywordlst}[1]{\text{\sffamily\bfseries #1}}
\newcommand{\id}[1]{\textsf{#1}} % identifiers

\newcommand{\mini}[1]{{\scriptscriptstyle #1}} % Smaller exponents

\definecolor{codeBlue}{rgb}{0.2,0.2,0.7}
\newcommand{\codeColour}{Blue}
\newcommand{\code}[1]{\textcolor{codeBlue}{#1}}
\newcommand{\Math}[1]{\textcolor{codeBlue}{$#1$}}
\newcommand{\MathCol}[1]{\textcolor{codeBlue}{#1}}

\newcommand{\rulename}[1]{\textsc{#1}}


% Meets and joins
\newcommand{\meet}{\sqcap}
\newcommand{\join}{\sqcup}
\newcommand{\bigJoin}{\bigsqcup}
\makeatletter % Replace with flipped \bigsqcup
\newcommand{\bigMeet}{%
    \mathop{%
        \mathpalette\@updown\bigsqcup
    }%
}
\newcommand*{\@updown}[2]{%
    \rotatebox[origin=c]{180}{$\m@th#1#2$}%
}
\makeatother

\newcommand{\hole}{\kern+0.7pt \fboxsep0pt\fboxrule=0.06em  \fbox{\phantom{$A^*A_.$}} \kern-0.1em}

% Constraints
\newcommand{\turnstile}{\vdash}

\newcommand{\set}[1]{\overline{#1}} % Overlined list/set notation
\newcommand{\cgen}{\ |\ } % Constraint gen arrow
\newcommand{\constr}[1]{\left\{#1\right\}} % Constraint set

\newcommand{\longsquig}{%
    \tikz \draw [->,
    line join=round,
    decorate, decoration={
        zigzag,
        segment length=4,
        amplitude=.9,post=lineto,
        post length=3pt
    }]  (0,0) -- (0.51,0);
}

% Environment definitions
\newenvironment{prooftreecustom}
{\begin{prooftree}\def\extraVskip{3pt}\def\labelSpacing{6pt}}
    {\end{prooftree}}

\newenvironment{infobox} % info boxes
  {\begin{tcolorbox}[colback=codeBlue!12!white,colframe=codeBlue!60!white,coltext=codeBlue!90!white,boxrule=.4mm]}
  {\end{tcolorbox}}

\newenvironment{simplebox} % Outline box
  {\begin{tcolorbox}[colback=white,colframe=black,sharp corners,boxrule=0.2mm]}
  {\end{tcolorbox}}


% \Alt for beamer
\makeatletter
\newcommand*\Alt{\alt{\Alt@branch0}{\Alt@branch1}}

\newcommand\Alt@branch[3]{%
  \begingroup
  \ifbool{mmode}{%
    \mathchoice{%
      \Alt@math#1{\displaystyle}{#2}{#3}%
    }{%
      \Alt@math#1{\textstyle}{#2}{#3}%
    }{%
      \Alt@math#1{\scriptstyle}{#2}{#3}%
    }{%
      \Alt@math#1{\scriptscriptstyle}{#2}{#3}%
    }%
  }{%
    \sbox0{#2}%
    \sbox1{#3}%
    \Alt@typeset#1%
  }%
  \endgroup
}

\newcommand\Alt@math[4]{%
  \sbox0{$#2{#3}\m@th$}%
  \sbox1{$#2{#4}\m@th$}%
  \Alt@typeset#1%
}

\newcommand\Alt@typeset[1]{%
  \ifnumcomp{\wd0}{>}{\wd1}{%
    \def\setwider   ##1##2{##2##1##2 0}%
    \def\setnarrower##1##2{##2##1##2 1}%
  }{%
    \def\setwider   ##1##2{##2##1##2 1}%
    \def\setnarrower##1##2{##2##1##2 0}%
  }%
  \sbox2{}%
  \sbox3{}%
  \setwider2{\wd}%
  \setwider2{\ht}%
  \setwider2{\dp}%
  \setnarrower3{\ht}%
  \setnarrower3{\dp}%
  \leavevmode
  \rlap{\usebox#1}%
  \usebox2%
  \usebox3%
}
\makeatother
