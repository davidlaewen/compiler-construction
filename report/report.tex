\documentclass{report}

\usepackage[a4paper]{geometry}

% Use nicer fonts
\usepackage[mono=false]{libertine}
\usepackage[libertine]{newtxmath}
\usepackage[varqu,scale=0.9]{zi4}

\usepackage{hyperref}

% Math stuff
\usepackage{amsmath}
\usepackage{stmaryrd}
\usepackage{mathtools}
\usepackage{mathpartir}
\usepackage{pftools} % Rule referencing
\usepackage{listings}

\usepackage[nameinlink,noabbrev,capitalise]{cleveref}

\usepackage[square]{natbib}
\bibliographystyle{ACM-Reference-Format}
\setcitestyle{acmauthoryear}

% Syntax highlighting (for Haskell)
\usepackage[outputdir=build]{minted}
\usemintedstyle{tango}
% \setminted{fontsize=\small}

\usepackage[most]{tcolorbox} % Colour boxes
\usepackage[color=orange!40]{todonotes} % To-do notes

% ACM colour definitions
\definecolor[named]{ACMBlue}{cmyk}{1,0.1,0,0.1}
\definecolor[named]{ACMYellow}{cmyk}{0,0.16,1,0}
\definecolor[named]{ACMOrange}{cmyk}{0,0.42,1,0.01}
\definecolor[named]{ACMRed}{cmyk}{0,0.90,0.86,0}
\definecolor[named]{ACMLightBlue}{cmyk}{0.49,0.01,0,0}
\definecolor[named]{ACMGreen}{cmyk}{0.20,0,1,0.19}
\definecolor[named]{ACMPurple}{cmyk}{0.55,1,0,0.15}
\definecolor[named]{ACMDarkBlue}{cmyk}{1,0.58,0,0.21}

% Other colour definitions
\usepackage{xcolor}

\definecolor{darkblue}{rgb}{0.2, 0.2, 0.6}
\definecolor{lightblue}{rgb}{0.3, 0.6, 0.7}
\definecolor{otherblue}{rgb}{0.0, 0.3, 0.4}
\definecolor{darkorange}{rgb}{0.5, 0.2, 0.0}
\definecolor{purple}{rgb}{0.5, 0.0, 0.5}
\definecolor{codeBlue}{rgb}{0.2,0.2,0.7}
\definecolor{codebg}{rgb}{0.9,0.9,0.9}

% Set up the hyperref package
\hypersetup{
  pdftitle = {SPL Compiler}, %   - title (PDF meta)
% pdfsubject={thesis subject},%   - subject (PDF meta)
  pdfauthor = {David Läwen}, %   - author (PDF meta)
  pdflang={en},
  unicode=true,
  colorlinks=true,           %   - colorize links?
  linktoc=all,               %   - sections and subsections linked
  breaklinks=false,          %   - allow line break inside links
% plainpages=false,
% pdfborder={0 0 0},
% bookmarksnumbered=true,
% bookmarksopen=true
% Hyperlink colours
  linkcolor = ACMPurple,
  citecolor = ACMPurple,
  urlcolor = ACMDarkBlue,
  filecolor = ACMDarkBlue
}

% Syntax highlighting for SPL
\lstdefinelanguage{SPL}{%
	alsoletter={0123456789ABCDEFGHIJKLMNOPQRSTUVWXYZabcdefghijklmnopqrstuvwxyz_+-*/\%=<>!\&|}
	morekeywords={while,else,if,print,return,isEmpty,fst,snd,hd,tl,Int,Char,Bool,Void,True,False,var},%
	sensitive=true,%
	morecomment=[l]{//},%
	morecomment=[n]{/*}{*/},%
	literate=%
		{->}{{$\rightarrow$}}2
		{>=}{{$\geq\:$}}1
		{<=}{{$\leq\:$}}1
		{==}{{$\equiv$}}1
}
\lstset{%
  mathescape,
	upquote=true,
	breakatwhitespace=false,
	breaklines=true,
	postbreak=\mbox{\textcolor{gray}{$\hookrightarrow$}\space},
	keepspaces=true,
	basicstyle=\tt,
	commentstyle=\sl,
	keywordstyle=\bf,
	stringstyle=\tt,
	showspaces=false,
	showstringspaces=false,
	showtabs=false,
	tabsize=4,
	basewidth=0.43em,
	columns=[c]fixed,
	texcl=true,
	captionpos=b
}

% Paragraph breaks
\newcommand{\br}{\\[2mm]}
\newcommand{\brbr}{\\[4mm]}

% To-do notes
\newcommand{\todobox}[2][]{\todo[inline,#1]{#2}}

% Grammar
\newcommand{\NT}[1]{\langle \textit{#1}\kern.7pt \rangle}
\newcommand{\ruleSep}{\quad\mid\quad}

% Inline syntax environments
\newcommand{\spl}[1]{\lstinline[language=SPL]|#1|}
\newcommand{\haskell}[1]{\mintinline{Haskell}|#1|}

\newcommand{\code}[1]{\texttt{#1}}

% SPL syntax
\newcommand{\Int}{\code{Int}}
\newcommand{\Bool}{\code{Bool}}
\newcommand{\Char}{\code{Char}}
\newcommand{\Void}{\code{Void}}
\newcommand{\TProd}[2]{\code{(}#1\code{,}#2\code{)}}
\newcommand{\TFun}[2]{#1 \to #2}
\newcommand{\TList}[1]{\code{[}#1\code{]}}

\newcommand{\Pair}[2]{\code{(}#1\code{,}#2\code{)}}
\newcommand{\Parens}[1]{\code{(}#1\code{)}}



% Meta notation
\newcommand{\set}[1]{\overline{#1}} % Overlined list/set notation
\newcommand{\fresh}{\mathit{fresh}}
\newcommand{\inst}{\mathit{inst}}
\newcommand{\unify}{\mathit{unify}}
\newcommand{\FV}{\mathit{FV}}

\newcommand{\eqdef}{\triangleq}

\newcommand{\bigO}{\mathcal{O}}




% Todo boxes
\newenvironment{todoenv}
	{\begin{tcolorbox}[colback=orange!12!white,colframe=orange,coltext=darkorange,boxrule=.4mm]}%
	{\end{tcolorbox}}


\author{%
	David~L\"awen\\
	\small\texttt{s1105105}\and
	Simcha~van~Collem\\
	\small\texttt{s1040283}
}
\date{\today}
\title{SPL compiler}

\begin{document}

\maketitle%

\tableofcontents%

\chapter{Introduction}
\begin{quote}
	\it
	This document is a mandatory skeleton for your report.
	It gives by no means an exhaustive list of suggestions for content.
	The report should be readable self-contained and readable without access to the source code.
\end{quote}
\begin{itemize}
	\item Motivate your language choice
	\item Introduce spl
	\item Give some nice examples.
	\item \ldots
\end{itemize}

\section{Language Choice}
For our compiler, we chose Haskell as our implementation language. Our choice
of Haskell was motivated in part by our prior experience working with Haskell,
we both took the Bachelor's course ``Functional Programming'', which also covers
monadic parser combinators as we use in our implementation.
Beyond the course on Haskell, ...

The approach of writing lexers and parsers using monadic function and
combinators is elegant and allows one to work at a high level abstraction.
In particular, the parser functions closely correspond to the rules of the
parser grammar that they express. This makes the parser and its constituent
parts easier to reason about.

For representing and manipulating abstract syntax trees, the algebraic data
types are also an obvious choice. Recursive functions over ADTs representing
AST variants as well as higher-order operations such as \emph{mapping} and
\emph{folding} allow many operators on ASTs to be expressed concisely.


\chapter{Lexing \& Parsing}
In this chapter we discuss the lexing and parsing phases of our compiler.
For both these phases we make use of the megaparsec library\footnote{\url{https://hackage.haskell.org/package/megaparsec}}.

\section{Lexing}
Our compiler has separate lexing and parsing phases.
This is mainly to make the work of the parser easier.
Firstly, the lexer takes care of skipping whitespace tokens.
This also includes skipping single and multiline comments.
Second, it also annotates all tokens with their position in the file.
The lexer returns a list of \mintinline{haskell}{Positioned Token},
where \mintinline{haskell}{Positioned a} is defined as follows:
\begin{minted}{haskell}
data Positioned a = Positioned
  { startPos :: SourcePos,
    endPos :: SourcePos,
    startOffset :: Int,
    tokenLength :: Int,
    tokenVal :: a
  }

data SourcePos = SourcePos
  { sourceName :: FilePath,
    sourceLine :: !Pos,
    sourceColumn :: !Pos
  }
\end{minted}
This will later make it possible to provide good error message to the user,
as they will be able to locate the error using the source positions.

The lexer has error recovery, as it is quite easy to implement.
There are basically two types of errors in our lexer.
Either the lexer encounters an invalid integer (e.g. \verb|123ab|),
or it encounters an unknown symbol.
In both cases we recover from this error, by skipping all characters until we
encounter a new token.

\section{Parsing}
\textbf{TODO:}

In order to write parser functions corresponding to the provided grammar of SPL,
we must transform the grammar in order to eliminate occurrences of left
recursion.
In particular, the rule
\[ \NT{Exp} \Coloneqq \NT{Exp}\ \NT{Op2}\ \NT{Exp} \]
is left-recursive and must be rewritten.
Additionally, when parsing expressions, we must consider the associativity and
precedence of the operators in the rules of our parser grammar in order to
obtain the desired AST.

We achieve this by creating a recursive descent parser with multiple levels,
corresponding to the precedence levels of the operators.
The parser thus begins at the lowest precedence level, featuring the Boolean
operators \code{\&\&} and \code{||}.
The leftmost non-terminal in the first rule is $\NT{Prop}$ and not $\NT{Expr}$
itself, allowing us to avoid infinite parsing loops caused by left recursion.

In order to still parse arbitrarily many applications of \code{\&\&} or
\code{||}, we use the pattern $((\code{\&\&} \mid \code{||})\ \NT{Prop})^*$,
hence we can parse any expression of the form $\dots \code{\&\&} \dots \code{||} \dots$,
and also simply a $\NT{Prop}$, since $*$ expresses zero or more occurrences of
a pattern.

For the following rules of the expression grammar, we proceed in much the same
manner: They consist of an expression matching the following rule, followed
by zero or more occurrences of a binary operator and another expression matching
the following rule.


In order to write an
expression such as
\[ \code{2 == 1+1 \&\& 2 != 3-2} \]
to be parsed as a conjunction of two (in)equalities without adding parentheses,
the precedence of \code{\&\&} must be lower than that of \code{==} and \code{!=}.



\begin{figure}[ht]
  \[
  \begin{array}{lcl}
		%
		\NT{Expr} & \Coloneqq & \NT{Prop}\ ((\code{\&\&} \mid \code{||})\ \NT{Prop})^* \br
		%
		\NT{Prop} & \Coloneqq & \NT{List}\ ((\code{==} \mid \code{!=} \mid \code{<} \mid \code{>} \mid \code{<=} \mid \code{>=})\ \NT{List})^* \br
		%
		\NT{List} & \Coloneqq & \NT{Form}\ (\code{:}\ \NT{Form})^* \br
		%
		\NT{Form} & \Coloneqq & \NT{Term}\ (( \code{+} \mid \code{-} )\ \NT{Term})^* \br
		%
		\NT{Term} & \Coloneqq & \NT{Val}\ (( \code{*} \mid \code{/} \mid \code{\%} )\ \NT{Val} )^* \br
		%
		\NT{Val} & \Coloneqq & \code{(}\ \NT{Expr}\ \code{)} \mid \code{(}\ \NT{Expr}\ \code{,}\ \NT{Expr}\ \code{)} \\
		& \mid & \NT{FunCall} \mid \NT{Id} \NT{Field} \mid \NT{Int} \mid \NT{Bool} \mid \NT{Char} \\
		& \mid & \code{[]} \mid \code{!}\ \NT{Val} \mid \code{-}\ \NT{Val}
		%
  \end{array}
  \]
  \caption{Parser grammar for expressions}
  \label{fig:refactored-grammar}
\end{figure}



\section{Abstract Syntax Tree}
\textbf{TODO: Update AST if we update code}
\begin{figure}
\begin{minted}[breaklines]{haskell}
data Program = Program [VarDecl] [FunDecl]
data VarDecl = VarDecl (Maybe Type) T.Text Expr
data FunDecl = FunDecl T.Text [T.Text] (Maybe Type) [VarDecl] [Stmt]

data Stmt = If Expr [Stmt] [Stmt]
          | While Expr [Stmt]
          | Assign Field Expr
          | FunCall T.Text [Expr]
          | Return (Maybe Expr)
          | GarbageS

data Type = IntT
          | BoolT
          | CharT
          | Prod Type Type
          | List Type
          | Void
          | Fun [Type] Type
          | TyVar T.Text
          | GarbageT

data Field = Ident T.Text
           | Head Field
           | Tail Field
           | Fst Field
           | Snd Field

data Expr = Field Field
          | Int Integer
          | Char Char
          | Bool Bool
          | UnOp UnaryOp Expr
          | BinOp BinaryOp Expr Expr
          | FunCallE T.Text [Expr]
          | EmptyList
          | Tuple Expr Expr

data UnaryOp = Not | Neg

data BinaryOp = Add | Sub | Mul | Div | Mod | Eq | Neq | Lt | Gt | Lteq | Gteq | And | Or | Cons
\end{minted}
	\caption{Abstract syntax tree}\label{fig:ast}
\end{figure}
Our syntax tree can found in Figure~\ref{fig:ast}.
Our top level node is \mintinline{haskell}{Program}.
A \mintinline{haskell}{Program} consists of a list of variable declarations,
and a list of function declarations.
One thing to note is that we have some garbage AST nodes, such as \mintinline{haskell}{GarbageS} and \mintinline{haskell}{GarbageT}.
These are used to support some sort of error recovery.
Using these garbage nodes we can still produce a valid AST,
such that the parser can recover and continue parsing.
It is essential that programs with these nodes will not be type checked.
After the parsing phase, the compiler should print the errors and stop compilation.
We can thus threat it as a compiler bug, if we come to a later phase and encounter these garbage tokens.

To model field lookups we use the \mintinline{haskell}{Field} datatype,
where a field \mintinline{haskell}{Snd (Fst "x")} should be interpreted as \lstinline[language=SPL]|x.fst.snd|.

We did not create a special data type for built-in functions like \lstinline[language=SPL]|print|, \lstinline[language=SPL]|isEmpty|, etc.
The main reason for this, is that we suspect that it will be easier to treat these as regular functions,
where there is already an implementation for them.
However, it may be the case that in the future we find that it is easier to create a special datatype for built-in functions,
which means we have to change it later.

We also did not add source position information to the AST yet.
We will do this whenever we find we need this information in the AST.

\begin{itemize}
	\item How did you design the Abstract Syntax Tree
	\item How does the parser work?
	\item How did you handle difficult things like fixity, associativity etc.
	\item Is there error handling? Recovery?
	\item Do you have a lexer and parser?
	\item How do they communicate?
	\item Problems?
	\item\ldots
\end{itemize}

\chapter{Analyses \& Typing}
\begin{itemize}
	\item New Abstract Syntax Tree? Decorate existing Abstract Syntax Tree?
	\item Error messages?
	\item Polymorphism? Inference? Overloading?
	\item Problems?
	\item\ldots
\end{itemize}

\chapter{Code Generation}
\begin{itemize}
	\item Compilation scheme?
	\item How is data represented? Lists tuples
	\item Semantics style, call-by-reference, call-by-value?
	\item How did you solve overloaded functions?
	\item Polymorphism?
	\item Printing?
	\item Problems?
	\item\ldots
\end{itemize}

\chapter{Extension}
Describe your extension in detail

\chapter{Conclusion}
What does work, what does not etc.

\section{Reflection}
\begin{itemize}
	\item What do you think of the project?
	\item How did it work out?
	\item How did you divide the work?
	\item Pitfalls?
	\item \ldots
\end{itemize}

\appendix
\chapter{Grammar}
Change the grammar to the one you actually used

\begin{verbatim}
SPL       = Decl+
Decl      = VarDecl
          | FunDecl
VarDecl   = ('var' | Type) id  '=' Exp ';'
FunDecl   = id '(' [ FArgs ] ')' [ '::' FunType ] '{' VarDecl* Stmt+ '}'
RetType   = Type
          | 'Void'
FunType   = [ FTypes ] '->' RetType
FTypes    = Type [ FTypes ]
Type      = BasicType
          | '(' Type ',' Type ')'
          | '[' Type ']'
          | id
BasicType = 'Int'
          | 'Bool'
          | 'Char'
FArgs     = [ FArgs ',' ] id
Stmt      = 'if' '(' Exp ')' '{' Stmt* '}' [ 'else' '{' Stmt* '}' ]
          | 'while' '(' Exp ')' '{' Stmt* '}'
          | id Field '=' Exp ';'
          | FunCall ';'
          | 'return' [ Exp ] ';'
Exp       = id Field
          | Exp Op2 Exp
          | Op1 Exp
          | int
          | char
          | 'False' | 'True'
          | '(' Exp ')'
          | FunCall
          | '[]'
          | '(' Exp ',' Exp ')'
Field     = [ Field ( '.' 'hd' | '.' 'tl' | '.' 'fst' | '.' 'snd' ) ]
FunCall   = id '(' [ ActArgs ] ')'
ActArgs   = Exp [ ',' ActArgs ]
Op2       = '+'  | '-' | '*' | '/'  | '%'
          | '==' | '<' | '>' | '<=' | '>=' | '!='
          | '&&' | '||'
          | ':'
Op1       = '!'  | '-'
int       = [ '-' ] digit+
id        = alpha ( '_' | alphaNum)*
\end{verbatim}

\end{document}
